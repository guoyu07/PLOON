\part{Inheritance}



\chapter{Introduction}

面向对象编程的关键思想是多态性(polymorphism),通过继承而相关联的类型就是多态类型,因此在许多情况下可以互换地使用派生类型或基类型地“许多形态”。

在C++语言中,多态性仅用于通过继承而相关联地类型的引用或指针。

\section{Overview}


面向对象编程基于三个基本概念:数据抽象、继承和动态绑定。

\begin{compactitem}
\item 数据抽象将任务组织成多个互相作用的松散耦合的软件组件;
\item 基于继承可以将类组织成一个层次结构,派生类继承基类的成员;
\item 动态绑定使编译器能够在运行时决定使用基类中定义的函数还是派生类中定义的函数。
\end{compactitem}


继承(inheritance)可以对类型之间的关系建模以共享公共的特性,从而可以使一个子类的实例存取与它的父类(超类)相关的数据和行为(方法)。

\begin{compactitem}
\item 派生类能够直接或间接地继承基类定义的成员;
\item 派生类无需修改就可以使用基类中与派生类的具体特性不相关的操作;
\item 派生类可以通过函数特化来重定义与派生类型相关的成员函数;
\item 派生类在从基类继承成员之外还可以定义更多自己的成员。
\end{compactitem}

买花的顾客Chris和花商Fred最终能联系到一起,我们希望花商展示特定行为的原因并不是因为他是一个花商,而仅仅是因为他是一个店主。例如,我们希望Fred通过收款来完成交易并以收据作为反馈,而且这种行为并不只是对花商有效,对面包师、杂货商、文具商、汽车经销商及其他商人都适用,这样就将花商的特定的行为与shopkeeper这个更一般的类别联系起来,并且由于Florists是店主的一个特殊表现形式,因此店主所具有的行为自动地也适用于花商这个子类。

继承在程序设计语言中的含义如下:

\begin{compactitem}
\item 子类所具有的数据和行为总是作为与其相关的父类的属性的扩展(extension)(即更大的集合),即子类具有父类的所有属性以及其他属性。
\item 子类是父类的更加特殊(或受限制)的形式,因此子类在某种程度上也是父类的收缩(contraction)。
\end{compactitem}

作为扩张和收缩之间的张力,继承是许多技术的内在来源,但同时也造成了许多对其使用的混乱。例如,C++中使用virtual指定的虚函数是基类希望派生类重新定义的,基类希望派生类继承的函数不能定义为虚函数。


继承总是可以传递的,这样类就可以继承各个级别的父类的特征。例如,如果Dog类是Labrador类的子类,并且Labrador类是Animal类的子类,那么Dog类将继承Labrador类和Animal类的属性。


\section{Relationship}




在检验两个概念是否为继承关系时存在一项规则,这项规则称为是一个(is-a)检验,其具体内容就是如果要检验概念A与概念B是否为继承关系时,可以尝试套用这个语句:
\[\mbox{A (n) A is a (n) B}\]

如果这个语句“听起来是对的”,那么这个继承关系很可能就是正确的。例如,下面的这些描述都是合理的说法:

\begin{compactenum}
\item 鸟是动物
\item 猫是哺乳动物
\item 苹果派是派
\item 文本窗口是窗口
\item 球是图形对象
\item 整数数组是数组
\end{compactenum}

另一方面,由于某种原因,下面的描述看起来就显得很奇怪,因此这些继承关系可能不成立。

\begin{compactenum}
\item 鸟是哺乳动物
\item 苹果派是苹果
\item 发动机是汽车
\item 整数数组是整数
\end{compactenum}

对于大多数情况,这种检验都能就继承技术使用的合理性给出一个可靠的判断。

\begin{compactitem}
\item 继承作为代码复用的手段。

子类可以继承父类的行为,因此子类中的一些代码就不必重新进行编写,当开发新程序时可以极大地减少其中的代码量。

\item 继承作为概念复用的手段。

在子类改写父类所定义的行为时,尽管此时并没有在父类和子类之间共享代码,但它们共享方法的定义。
\end{compactitem}

定义为public特征可以被类定义外部的代码所存取,而定义为private特征只能被类定义内部的代码所存取,继承引入了第三个标识protected。

在C++、C\#、Delphi、Ruby及其他语言中,定义为protected特征只能被类定义内部及其子类定义内部的代码所存取,因此定义为protected特征比定义为private特征更易于存取。



\begin{lstlisting}[language=C++]
class Parent{
private:
	int three;
protected:
	int two;
public:
	int one;
	Parent(){
		one = two = three = 42;
	}
	void inParent(){
		cout << one << two << three; /* all legal */ 
	}
};

class Child:public Parent{
	void inChild(){
		cout << one; // legal
		cout << two; // legal
		cout << three; // error - not legal
	}
};

void main(){
	Child c;
	cout << c.one; // legal
	cout << c.two; // error - not legal
	cout << c.three; // error - not legal
}
\end{lstlisting}

私有特征只能用于父类的内部,保护特征只能用于父类内部和子类内部,只有共有特征才能用于类定义外部。

Java语言也使用同样的关键字,但是对于Java语言,对保护特征的存取在声明这个类所在的包(package)内是合法的。


\section{Machanism}

按照继承机制可以将面向对象语言分为两类。

\begin{compactitem}
\item 每个类都必须继承于已存在的父类,例如Java、Smalltalk、Objective-C、C\#和Delphi Pascal。
\item 每个类不必继承于已存在的父类,例如C++和Apple Pascal等。
\end{compactitem}

对于那些要求所有的类都必须继承于已存在的类的语言来说,一个好处就是存在一个关于所有对象的根类,这个根类在Smalltalk和Objective-C语言中表示为Object,在Delphi Pascal语言中表示为TObject。

根类提供的任何行为都被所有的对象所继承,这样每一个对象都必然拥有一套公共的最小的函数集合。

一个大的继承树的缺点是它将所有类结合成一个紧密耦合的单元。而对于C++等语言的程序来说,由于可以拥有几个独立的继承层次,因此不必包含整个巨大的类库,每个程序只使用其中的一小部分。当然这也意味着,用户无法定义那种所有对象都必然包含的功能。

有时,也会从另外一种看待问题的视角来对语言进行划分。

根据语言是动态类型还是静态类型将其分为两类,动态语言的对象主要通过它所支持的消息进行描述,如果两个对象支持同一套消息,并且以类似的方式对其进行响应,那么实际上如果不考虑它们各自的类描述,将无法区分两个对象。在这种情况下,使所有对象都继承于一个公共基类的大部分行为将很有用处。

\begin{lstlisting}[language=C++]
// C++
class Wall : public GraphicalObject{
	...
};
// C#
class Wall : GraphicalObject{
	...
}
// CLOS
(defclass Wall(GraphicalObject))
// Java
class Wall extends GraphicalObject{
	...
}
// Object Pascal
type
	Wall = object(GraphicalObject)
		...
	end;
// Python
class Wall(GraphicalObject):
	def __init__(self):
		...
// Ruby
class Wall < GraphicalObject
	...
end
\end{lstlisting}


\chapter{Subclass}

研究静态类型语言中父类的数据类型与子类(或派生类)的数据类型之间的关系,可以发现以下的现象:

\begin{compactitem}
\item 子类的实例必须拥有父类的所有数据成员。
\item 子类的实例必须至少通过继承(如果不是显式地改写)实现父类所定义的所有功能(子类也可以定义新的功能)。
\item 在某种条件下,如果用子类实例来替换父类实例,那么将会发现子类实例可以完全模拟父类的行为,二者毫无差别。
\end{compactitem}

当检查关于继承不同的使用方式时,以上结论也并不总是有效。尽管如此,对于继承,以上结论仍然是较好的理想化描述,由此可以得出替换原则(principle of substitution)。

\begin{oopquote}
替换原则是指如果有A和B两个类,类B是类A的子类,那么在任何情况下都可以用类B来替换A,而外界毫无察觉。
\end{oopquote}

\chapter{Subtype}

子类型(subtype)是指符合替换原则的子类关系,以区别于一般的可能不符合替换原则的子类关系。



\begin{lstlisting}[language=C++]
procedure drawBoard;
var
	gptr : GraphicalObject;
	begin
		(* draw each graphical object *)
	gptr := listOfObject;
	while gptr <> nil do begin
		gptr.draw;
		gptr := gptr.link;
	end;
end;
\end{lstlisting}

全局变量listOfObject保持一个图形对象列表,图形对象可以是三种类型之一。变量gptr声明为一个图形对象,在执行循环的过程中,此变量所表示的对象将来自每个子类。

gptr有时表示球,有时表示洞口,有时表示桌壁。在每种情况下调用draw函数时,都会执行关于gptr当前值的正确的方法——而不是声明于类GraphicalObject中的方法。只有每个子类实现的功能都匹配于其父类所指定的功能,即子类必须同时也是子类型,代码才能够正确执行。

所有面向对象编程语言都支持替换原则,尽管某些语言在改写方法时需要附加的语法。

大多数语言都以一种非常直接的方式来支持替换原则,父类只包含从子类得来的值。不过,一个主要例外出现于C++语言中。

对于C++语言,只有指针和引用真正地支持替换原则,声明为值(不是指针)的变量不支持替换原则。

静态类型语言(例如C++、Object Pascal等)比动态类型语言(例如Smalltalk和Objective-C语言)更加强调替换原则。

静态类型语言倾向于通过类来表现对象,而动态类型语言则是通过行为来表现对象。


\begin{compactitem}
\item 在静态类型语言中,多态函数(可以通过多个不同类的对象来表示的函数)只有通过确保所有函数参数都是给定类的子类,才能实现特定的函数功能。
\item 在动态类型语言中,由于参数根本就没有类型化,因此同样的需求将变为参数必须能够对一组特定的消息进行响应。
\end{compactitem}

关于这种差别的一个例子就是需要一个以类Measureable的子类的实例作为参数的函数,与需要一个可以理解消息lessThan和equal的参数的函数进行比较。其中,前者通过对象所属的类来表现对象,后者通过对象的行为为表现对象,两种类型检查形式都应用于面向对象语言。

\section{Override}

子类有时为了避免继承父类的行为,需要对其进行改写(override)。

从语法上讲,这意味着子类将定义一个与父类有着相同名称且类型签名相同的方法。当改写与替换相结合时,就会出现这样一种情况:变量声明为一个类,它所包含的值来自于子类,与给定消息相对应的方法同时出现于父类和子类。

几乎在所有有这种情况的案例中,用户想要执行的方法都是来自于子类的方法,而忽略父类的同名方法。

对于许多面向对象语言来说,只要子类通过同一类型签名来改写父类的同名方法,自然便会发生所期望的行为。另一方面,还有一些语言需要用户说明是否允许这种替换,一般以关键字virtual来表明这一含义。

在C++语言中,必须在父类中使用virtual关键字(表明子类可能会发生改写行为,但不是一定发生这种行为)。实际上,C++语言中的虚拟改写比这里所描述的还要复杂。

\begin{lstlisting}[language=C++]
class GraphicalObject{
public:
	virtual void draw();
};
class Ball : public GraphicalObject{
public:
	virtual void draw(); // virtual optional here	
};
\end{lstlisting}

在Object Pascal语言中,可以在子类中使用virtual关键字(表明改写已经发生)。


\begin{lstlisting}[language=Pascal]
// Object Pascal
type
	GraphicalObject = object
		...
		procedure draw;
	end;

	Ball = object(GraphicalObject)
		...
		procedure draw; override;
	end;
\end{lstlisting}

另外,在C\#和Delphi语言中可能需要同时在父类和子类使用virtual关键字。



\begin{lstlisting}[language={[Sharp]C}]
// C#
class GraphicalObject{
	public virtual void draw(){...}
}

class Ball : GraphicalObject{
	public override void draw(){...}
}
\end{lstlisting}

在Delphi语言中需要分别使用virtual关键字来说明。



\begin{lstlisting}[language=Delphi]
// Delphi
type
	GraphicalObject = class(TObject)
		...
		procedure draw; virtual;
	end;

	Ball = class(GraphicalObject)
		...
		procedure draw; virtual;
	end;
\end{lstlisting}


\chapter{Classification}

使用继承的方式很多,而且于不同的类方法使用继承的方式不同,因此有些抽象方式有时只适用于特定的情况。

\begin{compactitem}
\item 特化(Specialization)——子类是父类的特殊情况,换言之,子类是父类的子类型。
\item 规范(Specification)——父类定义行为,但不实现行为,由子类来实现行为。
\item 构造(Construction)——子类使用父类提供的行为,但并不是父类的子类型。
\item 泛化(Generalization)——子类修改或改写父类的某些方法。
\item 扩展(Extension)——子类对父类增加新的功能,但并不改变任何继承于父类的行为。
\item 限制(Limitation)——子类限制使用继承于父类的某些行为。
\item 变体(Variance)——子类和父类之间都是对方的变体,并且可以任意选择两个对象之间的父类-子类关系。
\item 结合(Combination)——子类继承的特征来自于多个父类,这是一种多重继承。
\end{compactitem}


\section{Specialized Subclass}

继承和子类化最常用的场合可能就是特化。在特化子类化时,新类是父类的一种特殊形式,但是它符合父类的各种规定。因此,这种形式显然支持替换原则。这种分类方式(特化子类化)是使用继承最常用的思维方式,也是一个良好的设计所要努力达到的目标。

这里有一个特化子类化的实例。类Window提供一般的窗口操作(移动、改变大小、最小化等)。特化的子类TextEditWindow继承了窗口的操作,还提供了另外的功能,可以使窗口显示文本,并且用户可以对这些文本进行编辑。因为文本编辑窗口在总体上符合我们所要求的窗口的所有属性(这样,TextEditWindow窗口是Window的一个子类型,也是一个子类),因此我们将其作为特化子类化的一个实例。



\section{Specification Subclass}

另一个继承常用的场合就是保证使子类和父类维持同一个特定的公共接口——也就是说,它们实现同样的方法。父类是由一部分已实现的操作和一部分需推迟到子类来实现的操作组成的结合体。在父类与子类之间通常不存在接口的变化——子类仅实现父类所描述但却没有实现的行为。

实质上,规范子类化是特化子类化的一种特殊情况,只是这种子类化不是对已有类型的改进,而是实现了父类中未完成的抽象规范定义。在这种情况下,父类有时称为抽象规范类(abstract specification class)。

实现接口的类经常是继承的一种完全实现形式,然而,规范子类化也可以通过其他方式来实现。例如在模拟台球的实例中,由于类GraphicalObject没有实现关于描绘对象和响应对象撞击的方法,因此它是一个抽象类。其子类Ball、Wall和Hole通过规范子类化来实现这些方法。

一般来说,规范子类化的使用场合是,父类仅仅对行为进行定义,却没有对其进行实现,而由子类来实现这些行为。

\section{Construction Subclass}

子类一般通过父类就可以继承需要实现的几乎所有行为,只是需要对一些对应于接口的方法名称进行改变,或者以一种特定的方式对方法参数进行修改。即使父类与子类之间不符合这种是关系,上述的情况也依然存在。

例如,Smalltalk语言中的类层次实现了Dictionary数组的泛化。类似于数组,字典是一种由关键字-值对组成的集合,但是字典中的关键字可以为任意值。例如,可能用于编译器的符号表(symbol table)就可以看成是以符号名称为索引的字典。在这种字典中,符号表记录中的数值具有固定的格式(符号表输入记录),因此类SymbolTable可以作为类Dictionary的子类,并且定义了一些用于符号表的新方法。另外一个例子是关于基类的一套数据抽象,它提供了列表方法。

对于这两个实例,由于我们从不会考虑在某种条件下,用子类实例来替换正在使用的父类的实例,因此子类并不是父类的更特殊的形式。

构造子类化通常用于创建对持久存储系统的二进制文件进行写操作的类。父类仅仅实现对原始二进制数据的写操作。子类主要用于存储每种结构。子类通过存储过程,使用父类型的行为来完成数据类型的实际存储。

这样的例子说明了分类之间的模糊性,如果子类使用不同名称的方法来实现存储,我们就称之为构造子类化。另一方面,如果子类使用与父类相同名称的方法来实现存储,我们就称之为构造子类化。

\begin{lstlisting}[language=C++]
class Storable{
	void writeByte(unsigned char);
};
class StoreMyStruct : public Storable{
	void writeStruct(MyStruct & aStruct);
};
\end{lstlisting}

构造子类化经常违反替换原则(形成的子类并不是子类型),因此静态类型语言不支持构造子类化。

另一方面,构造子类化通常可以快速简单地开发新的数据抽象,因此在动态类型语言中得到了广泛的应用。例如,在Smalltalk语言标准库中有许多通过构造子类化形成的实例。

C++语言提供了一种有趣的机制——私有继承,通过它可以在不违反替换原则的前提下实现构造子类化。

\section{Generalization Subclass}

在某种程度上,通过继承进行的泛化子类化与特化子类化恰好相反。在这种情况下,子类将父类的行为进行扩展,建立了一种更泛化的对象。泛化子类化的使用场合通常是我们打算基于已存在的类来建立一种不想修改或者不能修改的类。

从另一个角度来考虑一下图形显示系统,在这里,类Window定义为在简单的黑白背景下进行显示。可以建立一个子类型ColoredWindow,通过增加一个存储颜色的数据字段,我们可以改写继承之后的窗口,使之显示相应的背景颜色来代替原来的黑白颜色。

泛化子类化通常用于基于数据值的整体设计,其次才是基于行为的设计。这可以通过彩色窗口这个实例加以说明,彩色窗口就包含了简单窗口所不需要的数据字段。

作为规则,泛化子类化应该避免转换成为类型层次和使用特化子类化。然而,这并不总是容易做到的。



\section{Extension Subclass}

泛化子类化是对一个对象已存在的功能进行修改或扩展,而扩展子类化则是对对象增加新功能。扩展子类化与泛化子类化之间的区别在于,后者必须至少改写来自父类的一个方法,而且子类的功能需要与父类紧密联系,而扩展子类化只是对父类增加新的方法,并且子类的功能与父类的联系并不是那么紧密。

一个扩展子类化的实例就是继承于Set类用于存储字符串值的StringSet类。这个类将提供关于字符串操作的附加方法——例如,“通过前缀进行查找”的功能将返回关于所有元素的集合的一个子集,这个子集中的字符串都以指定的前缀开头。这些操作对于子类都很有意义,但对于父类却关系不大。

对于扩展子类化,父类的功能是可用的,而且是未用的,扩展子类化并不违反替换原则,因此扩展子类化中的子类也是子类型。

\section{Limitation Subclass}

限制子类化具体是指子类的行为少于或限制于父类。与泛化子类化一样,限制子类化经常用于这种场合,就是用户需要创建新类,这种新类继承于已存在的不应该或不能进行修改的类。

例如,一个已存在的类库提供一个双向队列(deque)数据结构,可以从队列的两端添加或删除元素。但是,现在用户希望写一个堆栈类,需要的功能是只能从堆栈的一端添加或删除元素。

与构造子类化相近,用户可以通过继承已存在的Deque类来创建Stack类,并且修改或者改写那些不符合要求的方法,使其在被调用时产生错误消息。这些方法通过削弱父类的功能改写了父类已存在的方法,这就是限制子类化。

由于限制子类化显然违反了替换原则,通过它来的创建的子类不是子类型,因此应该尽可能避免使用限制子类化。


\section{Variance Subclass}



在两个或多个类需要实现类似的功能,但它们的抽象概念之间似乎并不存在层次关系的情况下,可以使用变体子类化。例如,用来控制鼠标的代码可能与用来控制图形输入板的代码几乎完全相同。但是,在概念上,没有理由让其中任何一个类来作为另外一个类的子类。因此可以选择其中任何一个类作为父类,另外一个类继承于这个父类,并改写与设备相关的代码。



但是,通常使用的更好的替代方法是将两个类的公共代码提炼成一个抽象类(比如类PointingDevice),并且让这两个类都继承于这个抽象类。与泛化子类化一样,当基于已存在的类创建新类时,就无法使用这种方法了。

\section{Combination Subclass}

结合子类化使用的场合是,需要一个子类,可以表示两个或更多的父类的特征的结合。例如,一个助教,既有教师的特征,也有学生的特征,因此在逻辑上可以表示成任何一个。

通常,可以继承于两个或多个父类的能力称为多重继承(multiple inheritance)。

\chapter{Features}

\section{Reusability}


当继承另外一个类的行为时,提供这些行为的代码将不需要重新编写。在未出现继承以前程序员需要重写那些已经写过很多次的代码,——例如,寻找符合特定模式的字符串,或者为表格增加新元素等常用操作。通过面向对象技术,这些函数可以只编写一次就获得复用。

可复用代码的另外一个优点就是增加了程序的可靠性(代码复用的比例越高,发现错误的几率就越大),因此降低了由于多个用户共享同一代码所产生的维护成本。

\section{Collective}

代码共享涉及面向对象技术的各个方面,其中的一个层面就是,许多用户或项目可以使用同一个类(Brad Cox称之为软件集成电路,与硬件设计中的集成电路相对应)。

另外一种代码共享就是一个程序员在开发项目的一部分时,通过继承一个父类来创建两个或多个子类。例如,Set和Array都可以看作是Collection的一种形式。此时,两种或多种类型的对象共享继承的代码,这种代码只需编写一次,并且对于最终程序大小的影响也只有一次。


\section{Consistency}

当两个或更多的类继承于同一个超类时,我们可以确保它们所继承的行为是相同的。这样,就更容易保证相似的对象具有相似的接口,并且用户可以避免将一些表面相似但行为和相互作用方式截然不同的对象组成一个混乱的集合。


\section{Component}

继承为用户提供了一种构造可复用软件组件的手段,以实现几乎不需要编写多少代码就可以开发出新的应用程序的目标。在目前的软件工程中,商用类库和专用系统类库等的使用已经很普遍。

\section{Prototype}

当软件系统主要是由一些可复用的组件构成时,主要的开发时间将集中在了解系统的不常用的和新的组成部分。这样,软件系统就可以快速容易地完成。这种编程形式称为快速原型法编程或探索编程。

在快速开发完原型系统后,用户对其进行实验,然后基于第一个系统再创建第二个系统,再进行新系统的实验等,不断地进行多次的迭代实验。当开始一个软件项目时,如果对软件系统的目标和需求只能进行模糊的把握时,这种开发方式将非常有用。


\section{Polymorphism}

对于传统的软件开发方式,虽然在软件设计时可能是自上而下的,但是在编码时通常都是自下而上的,即先要编写好低级例程,在此之上再建立稍高一级的抽象,然后再建立更高级的抽象。这个过程就像构建一堵墙,每块砖都必须放置在已经砌好的砖上。

通常,代码的可移植性随着抽象级别的提高而降低,也就是说最低级别的例程可以用于多个不同的项目,下一级别的抽象可能还会得到复用,但是更高级别抽象的例程将与特定的应用程序紧密地结合在一起。

\begin{compactitem}
\item 低级别抽象的程序代码段可以用于新系统,并且一般仍然有意义。
\item 更高级别的组件则只能构建在特定的低级别单元上时才有意义(由于声明或数据依赖关系)。
\end{compactitem}

编程语言中的多态允许程序员创建高级别的可复用组件,通过对这些组件进行裁剪,可以适应各种不同的应用程序的变化。

使用可复用组件的程序员只需了解组件的性质和接口,而不必了解用于实现组件的技术的详细信息,这样可以减少软件系统的相互关联,而这种相互关联的特性是导致传统软件变得复杂的一个主要原因。


\chapter{Disadvantage}

\section{Efficiency}

作为通用的软件工具,其运行速度几乎不可能比专用软件系统快。这样,继承方法由于必须处理相应的子类,因此程序运行速度通常比专用代码慢。

但是,一味的关心程序运行效率是片面的,Bill Wulf提出了关于效率重要性的一些很好的见解,“越来越多的过错被归结到程序的运行效率上(没有实现相应的效率),而忽视了其他的原因,……”。

\begin{compactitem}
\item 首先,这种速度的差异通常很小。
\item 其次,执行速度的下降通常会通过软件开发速度的提高得以平衡。
\item 最后,大多数程序员实际上很少会了解他们所使用的程序的执行时间主要花费在哪些地方。
\end{compactitem}

在项目开发的早期花费大量的时间来考虑程序的运行效率是不明智的,更好的做法是开发完成系统后投入运行,对其进行监控,找到程序的执行时间主要耗费在什么地方,再对这部分代码进行改进。

\section{Capacity}

除了为特定的项目构造的系统之外,任何软件库的使用都会增加程序的大小。

不过,这种情况随着目前硬件的发展,软件的大小变得越来越不重要了。现在,控制开发的费用和快速构建高质量、无错误代码比限制程序的大小变得更加重要。

\section{Message}

消息传递是一项在本质上比过程调用代价更高的操作,但是与整体执行速度一样,过多地考虑消息传递的开销通常是一种大头不算小头算的做法。因为考虑消息传递所增加的代价通常都很小——可能也就是增加几条汇编语言指令,并且运行时间提高10\%。

运行时间随着语言的不同而不同。

\begin{compactitem}
\item 对于动态绑定语言(例如Smalltalk语言),消息传递的开销要高些;
\item 对于静态绑定语言(例如C++语言),相应的开销要低些。
\end{compactitem}

消息传递的确增加了开销,但是这种开销需要与面向对象技术所带来的优点进行权衡。


不同的语言(尤其是C++语言)为用户提供了许多用来减少消息传递开销的可选方案,其中包括消除消息传递中的多态机制(在C++语言中,限定通过类名称进行成员函数调用)和扩展内联过程。

类似地,Delphi Pascal程序员可以选择使用运行时检测机制的动态(dynamic)方法,或使用更快一些的技术的虚拟(virtual)方法。动态方法要慢一些,但是需要的空间更少一些。

\section{Complexity}

正确合理地使用面向对象技术可以降低软件的复杂度,但是实际开发过程中,对继承的过度使用通常会使软件由一种形式的复杂转向另一种形式的复杂。

为了真正理解使用继承的程序的流程控制,需要对程序的整个继承图进行多次的遍历扫描,这就是通常所说的yo-yo(溜溜球)问题。

当声明某一子类继承于一个父类时,父类的代码将不需要重新编写,这样继承可以作为一种代码复用的强大的手段,但是这并不是使用继承的唯一原因。

在抽象的意义上,子类是父类所属范畴的一个代表,因此在那些希望使用父类实例的情况下,使用子类的实例将同样有意义,这也称为替换原则。但是,这只是一种理想情况,并不是所有的继承都符合这种理想的行为。

关于替换原则思想的早期讨论主要是由Barbara Liskov和John Guttag完成的,因此替换原则有时也称为Liskov替换原则。


继承既可用于代码复用,又可用于概念复用。这种同一特征可以用于两种目的的事实引发了关于面向对象编程语言的很多讨论,于是有提议将这两项任务分离。

\begin{compactitem}
\item 通过类继承来实现代码复用;
\item 通过接口继承(例如Java语言)来实现替换(概念复用)。
\end{compactitem}

上述方法虽然具有理论基础,但在实践中却使编程任务变得更加复杂,没有得到广泛的应用。

在主要的编程语言中,由于Java语言支持类和接口的概念,因此将继承的这两个目的分离得最彻底。但是由于它仍然允许对于类值的替换,因此对这两个主题还是产生了一定的混淆。

\begin{lstlisting}[language=C++]

\end{lstlisting}





\begin{lstlisting}[language=C++]

\end{lstlisting}





\begin{lstlisting}[language=C++]

\end{lstlisting}




\begin{lstlisting}[language=C++]

\end{lstlisting}





\begin{lstlisting}[language=C++]

\end{lstlisting}





\begin{lstlisting}[language=C++]

\end{lstlisting}





\begin{lstlisting}[language=C++]

\end{lstlisting}





\begin{lstlisting}[language=C++]

\end{lstlisting}





\begin{lstlisting}[language=C++]

\end{lstlisting}




\begin{lstlisting}[language=C++]

\end{lstlisting}





\begin{lstlisting}[language=C++]

\end{lstlisting}






\begin{lstlisting}[language=C++]

\end{lstlisting}





\begin{lstlisting}[language=C++]

\end{lstlisting}





\begin{lstlisting}[language=C++]

\end{lstlisting}






\begin{lstlisting}[language=C++]

\end{lstlisting}





\begin{lstlisting}[language=C++]

\end{lstlisting}





\begin{lstlisting}[language=C++]

\end{lstlisting}




\begin{lstlisting}[language=C++]

\end{lstlisting}





\begin{lstlisting}[language=C++]

\end{lstlisting}





\begin{lstlisting}[language=C++]

\end{lstlisting}





\begin{lstlisting}[language=C++]

\end{lstlisting}





\begin{lstlisting}[language=C++]

\end{lstlisting}





\begin{lstlisting}[language=C++]

\end{lstlisting}





\begin{lstlisting}[language=C++]

\end{lstlisting}





\begin{lstlisting}[language=C++]

\end{lstlisting}