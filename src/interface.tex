\part{Interface}




\chapter{Introduction}

接口泛指实体把自己提供给外界的一种抽象化物(可以为另一实体),用以由内部操作分离出外部沟通方法,使其能被修改内部而不影响外界其他实体与其互动的方式。例如,面向对象程序设计通过接口来多重抽象,从而对组件的功能进行合理的抽象。

实际上,程序组件的接口会被存取到的事物的种类可以包括常量、公共变量、数据类型、过程类型、异常和方法签名等。

接口与其实现是分离的,其他组件只能通过接口来进行交互,从而替换接口的实现不会影响接口的用户。

\begin{quote}
\emph{在面向对象的程序设计中,里氏替换原则(Liskov Substitution principle)是对子类型的特别定义,其内容可以描述为: 派生类(子类)对象能够替换其基类(超类)对象被使用。\footnote{Barbara Liskov与Jeannette Wing在1994年发表论文并提出Liskov代换原则:\newline Let $q(x)$ be a property provable about objects $x$ of type $T$. Then $q(y)$ should be true for objects y of type $S$ where $S$ is a subtype of $T$.}}
\end{quote}

在面向对象编程中,接口通常定义为一些方法的集合,对对象的属性的访问通常通过属性存取函数来进行。


接口投入使用之后就不应该被修改。如果接口的实现模块提供了新的功能,而想在其他模块中调用这个功能,那么需要定义新的部份而不是修改现存的接口。

尽管接口的定义没有强制的标准,但是一些标准的COM接口的应用十分广泛,例如IUnknown和IDispatch。

在面向对象程式设计中,一些支持动态语言的模块实现了IDispatch来支持在运行时“发现”对象提供的函数、方法和事件(通常称为自动化),但是这个通过IDispatch来做代理的方法使得程式性能有所降低。



\section{Overview}



接口是为某种行为定义了一份协议,但是不提供任何实现。下面就是一个接口的例子,描述了对象可以从输入/输出流中进行读取和写入操作。



\begin{lstlisting}[language=Java]
public interface Storing{
	void writeOut(Stream s);
	void readFrom(Stream s);
}
\end{lstlisting}

类和接口都定义了一种新类型,这意味着可以仅仅通过接口的名称来声明变量。


\begin{lstlisting}[language=Java]
Storing storableValue;
\end{lstlisting}

类可以表明它将要实现哪些接口定义的协议,而且类的实例可以赋值给接口类型变量。


\begin{lstlisting}[language=Java]
public class BitImage implements Storing{
	void writeOut(Stream s){
	//...
	}
	void readFrom(Stream s){
	//...
	}
}
storableValue = new BitImage();
\end{lstlisting}

接口的使用和继承的概念十分相似,但是接口只描述行为,不提供具体实现。

从某种程度上来说,接口是一种近似于类的实体,因此Java和C++中的接口的概念和类的概念是密切相关的。


\begin{lstlisting}[language=Java]

\end{lstlisting}





\begin{lstlisting}[language=Java]

\end{lstlisting}





\begin{lstlisting}[language=Java]

\end{lstlisting}




\subsection{C++}





\subsection{Java}



\section{Abstract Class}


在面向对象编程语言中,接口与类有一些一致的地方,接口可以继承于其他接口,也可以继承于多个父接口。


虽然继承类的规范和实现接口的规范并不完全相同,但它们非常相似,因此将使用继承这一术语来描述这两种行为。

有些面向对象语言支持一种称为抽象方法(abstract method)的术语,它是一种介于类和接口之间的概念。例如,在Java和C\#语言中,类可以使用abstract关键字定义一个或多个方法,但不对这些方法进行实现。

在创建类的实例之前,子类必须实现父类的每一个抽象方法,因此抽象方法的行为由父类进行指定,但是必须由子类来提供这些行为的实现。



\begin{lstlisting}[language=Java]
abstract class Window{
	...
	abstract public void paint(); // draw contents of window
}
\end{lstlisting}

或者,也可以将整个类命名为抽象类,而不管这个类是否包含抽象方法。

在使用时,禁止创建关于抽象类的实例,抽象类只能用于继承,从而作为其他类的父类。

C++语言使用纯虚方法(pure virtual method)来表示抽象方法这个概念,并且通过赋值操作符来表示。

\begin{lstlisting}[language=C++]
class Window{
public:
	...
	virtual void paint() = 0; // assignment makes it pure virtual
};
\end{lstlisting}

类可以同时包含抽象(或纯虚)方法和非抽象方法,所有方法都声明为抽象(或纯虚)方法的类相当于Java语言中的接口的概念。


即使一种语言不显式支持抽象方法这个概念,也可以通过其他方法进行模拟。例如,在Smalltalk语言中,用户经常定义一种调用时产生错误的方法,它将被子类中的方法所覆盖。

\begin{lstlisting}[language=Java]
writeTo: stream
'$\uparrow$' self error: 'subclass must override writeTo'
\end{lstlisting}

实际上,并不能阻止创建该方法所处的类的实例,因此这种方法与真正的抽象方法并不相同。尽管可以创建这样的类实例,但是由于程序在调用方法时会失败,因此很容易检测这种错误。









\begin{lstlisting}[language=Java]

\end{lstlisting}





\begin{lstlisting}[language=Java]

\end{lstlisting}





\begin{lstlisting}[language=Java]

\end{lstlisting}









\begin{lstlisting}[language=Java]

\end{lstlisting}





\begin{lstlisting}[language=Java]

\end{lstlisting}





\begin{lstlisting}[language=Java]

\end{lstlisting}
