\part{Interface}




\chapter{Introduction}

接口泛指实体把自己提供给外界的一种抽象化物(可以为另一实体),用以由内部操作分离出外部沟通方法,使其能被修改内部而不影响外界其他实体与其互动的方式。例如,面向对象程序设计通过接口来多重抽象,从而对组件的功能进行合理的抽象。

实际上,程序组件的接口会被存取到的事物的种类可以包括常量、公共变量、数据类型、过程类型、异常和方法签名等。

接口与其实现是分离的,其他组件只能通过接口来进行交互,从而替换接口的实现不会影响接口的用户。

\begin{quote}
\emph{在面向对象的程序设计中,里氏替换原则(Liskov Substitution principle)是对子类型的特别定义,其内容可以描述为: 派生类(子类)对象能够替换其基类(超类)对象被使用。\footnote{Barbara Liskov与Jeannette Wing在1994年发表论文并提出Liskov代换原则:\newline Let $q(x)$ be a property provable about objects $x$ of type $T$. Then $q(y)$ should be true for objects y of type $S$ where $S$ is a subtype of $T$.}}
\end{quote}

在面向对象编程中,接口通常定义为一些方法的集合,对对象的属性的访问通常通过属性存取函数来进行。


接口投入使用之后就不应该被修改。如果接口的实现模块提供了新的功能,而想在其他模块中调用这个功能,那么需要定义新的部份而不是修改现存的接口。

尽管接口的定义没有强制的标准,但是一些标准的COM接口的应用十分广泛,例如IUnknown和IDispatch。

在面向对象程式设计中,一些支持动态语言的模块实现了IDispatch来支持在运行时“发现”对象提供的函数、方法和事件(通常称为自动化),但是这个通过IDispatch来做代理的方法使得程式性能有所降低。



\section{Overview}



接口是为某种行为定义了一份协议,但是不提供任何实现。下面就是一个接口的例子,描述了对象可以从输入/输出流中进行读取和写入操作。



\begin{lstlisting}[language=Java]
public interface Storing{
	void writeOut(Stream s);
	void readFrom(Stream s);
}
\end{lstlisting}

类和接口都定义了一种新类型,这意味着可以仅仅通过接口的名称来声明变量。


\begin{lstlisting}[language=Java]
Storing storableValue;
\end{lstlisting}

类可以表明它将要实现哪些接口定义的协议,而且类的实例可以赋值给接口类型变量。


\begin{lstlisting}[language=Java]
public class BitImage implements Storing{
	void writeOut(Stream s){
	//...
	}
	void readFrom(Stream s){
	//...
	}
}
storableValue = new BitImage();
\end{lstlisting}

接口的使用和继承的概念十分相似,但是接口只描述行为,不提供具体实现。

从某种程度上来说,接口是一种近似于类的实体,因此Java和C++中的接口的概念和类的概念是密切相关的。



\begin{lstlisting}[language=Java]

\end{lstlisting}





\begin{lstlisting}[language=Java]

\end{lstlisting}





\begin{lstlisting}[language=Java]

\end{lstlisting}




\subsection{C++}





\subsection{Java}




